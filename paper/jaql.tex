\section{Jaql}

Jaql~\cite{jaqlWebsite} is a high-level query language for the JavaScript Object Notation (JSON).
It is able to run on Hadoop and break most requests down to Map/Reduce tasks.
Jaql heavily borrowed from SQL, XQuery, LISP, PigLatin, JavaScript and Unix Pipes.~\cite{jaqlOverview}

Developed mainly inside IBM and with an quite mailinglist Jaql currently faces
a serious lack of documentation and community. The documentation found online in
most cases is outdated and incomplete. In addition to this, Jaql was undergoing
major changes in the time of writing.

Even though available, there is no need to write in a strict map/reduce pattern in
jaql. At the point of execution (i.e. the end of a query statement) Jaql transforms the
parsed statement into another, equivalent but optimized statement. This step is
comparable to a query optimizer in modern database management systems. The optimized
query can in turn be transformed back to Jaql code, which is useful for debugging.

Jaql can be extended with user-defined functions, that can be written either in Java or
in Jaql itself. However, it is not possible to use Jaql as a general-purpose programming
language, as it is not Turing-complete: It lacks both recursion and a universal loop
function. A short attempt to implement a Y combinator in Jaql only led to system crashes.
Since this is none of Jaql's goals, it is unlikely to change in the near future.

The current Jaql implementation features three modes to run in: In stand-alone mode hadoop
is not used at all and the jobs are not split in map and reduce tasks. When using jaql
with hadoop you can either use a mini-cluster, which is managed by jaql an runs all tasks
in the same process (with one thread per map/reduce task), or you can use an existing,
currently running hadoop cluster in which the node you run jaql on takes part in.

One of the main features of Jaql is the ability to read from and write to different
storage types, currently the local file system, the Hadoop Distributed File System,
HBase (Hadoop's BigTable equivalent) and HTTP. Since Jaql is based on JSON, reading
from HTTP allows easy integration of web services like Wikipedia or Flickr, that offer
a JSON API.

\begin{lstlisting}[language=jaql,caption=A sample Jaql query,float,label=jaqlsample]
$source = "http://server/api.php?format=json\&query=foo";
read(http($source)) 
  -> transform each $entry $entry["name"]
  -> write(hdfs("output.dat"));
\end{lstlisting}

A typical jaql request can be seen in listing \ref{jaqlsample}. Like Pig, Jaql ships with an
interactive shell (read-eval-print loop). 