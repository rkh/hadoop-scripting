\section{Programming with Jaql}

\begin{lstlisting}[language=jaql,caption=A sample Jaql query,label=jaqlsample]
$source = "http://server/api.php?format=json\&query=foo";
read(http($source)) 
  -> transform each $["name"]
  -> write(hdfs("output.dat"));
\end{lstlisting}

One of the main features of Jaql is the ability to read from and write to different
storage types, currently the local file system, the Hadoop Distributed File System,
HBase and HTTP. Since Jaql is based on JSON, reading
from HTTP allows easy integration of web services like Wikipedia or Flickr, that offer
a JSON API.

A typical Jaql request can be seen in listing \ref{jaqlsample}. Jaql ships with an
interactive shell (read-eval-print loop). 

Jaql features two ways of passing parameters to a function. You can write the parameters
in brackets after the function name, as known from most other programming languages, i.e.
\lstinline[language=jaql]!foo("bar")!. However, Jaql also features a new pipe syntax, more
consistent with the way built in Jaql statements are written. The value you ``pipe into''
a function will be past as the first argument, therefore \lstinline[language=jaql]!foo("bar", "baz")!
is equivalent to \lstinline[language=jaql]!"bar" -> foo("baz")!.

All values, either arguments or return values, are JSON objects or Jaql functions.
Every function in Jaql returns a value. Jaql offers advanced functionality for querying, transforming
and aggregating those values. An example can be seen in listing \ref{jaqlwc}. Note that
operations like {\tt expand} or {\tt group by} will loop through all element of the given array
and potentially split this operation on multiple Map/Reduce tasks. Within a loop {\tt\$} is referring
to the current element.

\begin{lstlisting}[language=jaql,caption=Wordcount in Jaql,label=jaqlwc]
read($src)
  -> expand splitArr($, " ")
  -> group by $w = ($) into [$w, count($)];
\end{lstlisting}

A statement is terminated by a semicolon and will at that point be optimized, as mentioned earlier.
However, instead of executing the statement, Jaql is able to return the optimized Jaql code by calling
{\tt explain} on the statement.