\section{Pig}

Pig~\cite{pigWebsite}  is a high level scripting language for data transformation. It is a Hadoop Subproject and in the Apache Incubator since 2007. Just as Hadoop is is mainly developed by Yahoo!, where in early 2009 30 \% of all Map/Reduce jobs are already implemented via Pig.~\cite{pig30percent}

It has an active community and a growing ecosystem.

The development of Pig took three key aspects into account. \\
The ease of programming was the major goal enabling the user to write powerful scripts that are easy to write, read and maintain against very large data sets .~\cite{pigWebsite}

Optimisation was another goal. When writing a Pig script, it is not needed for the programmer to think within the Map/Reduce paradigm since Pig handles the transformation of the script to the particular Map/Reduce jobs. This also offers opportunities for automatic optimisations made by the Pig compiler. A good explanation about how Pig parses and compiles a Pig script can be found in the paper ``Pig Latin: A Not-So-Foreign Language for Data Processing'' Chapter 4.~\cite{pigNotForeign}

Pig features extensibility achieved by user-defined functions which are programmed in Java against the Pig interfaces and may be called within Pig Latin. Thus Pig has the same feature set as Java Hadoop.

Pig uses the scripting language Pig Latin~\cite{pigManual}. The syntax looks similar to SQL (which may be integrated into Pig additionally~\cite{pigSql}), albeit Pig Latin is a data transformation language and therefore more similar to the database query optimiser than to a SQL statement.

A typical Pig Latin line of code looks like listing \ref{pigsample}.

\begin{lstlisting}[language=pig,caption=A typical Pig line of code,label=pigsample]
ordered = ORDER words;
\end{lstlisting}

It is the data transformation of one set (``words'') with an operation (``ORDER'') into a new set (``ordered'').

Most often Pig Latin Scripts follow this listings structure \ref{pigstructure}.

\begin{lstlisting}[language=pig,caption=Pig Latin Script Structure ,label=pigstructure]
Load Data -> Filter Data -> Group Data -> Output Data
\end{lstlisting}
                                                     
Pig Latin Scripts often do not consist of more than 10 lines of code, whereas user-defined functions in Java are more complex and may need more than 100 lines of code.
                                                                                                               